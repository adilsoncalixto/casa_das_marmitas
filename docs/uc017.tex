\chapter{Exibir Produto - UC017} \label{uc017}

\section{Breve descrição}

Após a pesquisa ou o cadastro de uma marmita, o sistema carregará um formulário com os dados da mesma.

\section{Atores}

\begin{enumerate}
	\item Funcionário da empresa, Casa das Marmitas.
\end{enumerate}

\section{Pré-condições}

\begin{enumerate}
	\item O funcionário deverá possuir login e senha de acesso autenticados pelo sistema.
	\item O funcionário deverá ter executado, anteriormente, o caso de uso \nameref{uc005} ou \nameref{uc012}.
\end{enumerate}

\section{Fluxo de eventos}

\subsection{Fluxo básico}

\begin{enumerate}[label=P\arabic*]
	\item O sistema apresenta a tela \textbf{Exibir Marmita} com os campos \ref{uc017_rn:1}. \label{uc017_p:1}
	\item O sistema habilita a opção \textbf{Alterar}. \label{uc017_p:2}\ref{uc017_a:1} 
	\item O sistema habilita a opção \textbf{Excluir}. \label{uc017_p:3}\ref{uc017_a:2}
	\item Esse caso de uso é encerrado. \label{uc017_p:4}\ref{uc017_a:3} \ref{uc017_a:4}
\end{enumerate}

\subsection{Fluxos alternativos}

\begin{enumerate}[label=A\arabic*]
	\item Alternativa ao passo \ref{uc017_p:2} - O funcionário não tem permissão de alterar os dados da marmita \label{uc017_a:1}
	\begin{enumerate}[label*=.\arabic*]
		\item O sistema desabilita a opção \textbf{Alterar}.
		\item O sistema vai para o próximo passo.
	\end{enumerate}
	
	\item Alternativa ao passo \ref{uc017_p:3} - O funcionário não tem permissão de excluir a marmita \label{uc017_a:2}
	\begin{enumerate}[label*=.\arabic*]
		\item O sistema desabilita a opção \textbf{Excluir}.
		\item O sistema vai para o próximo passo.
	\end{enumerate}
	
	\item Alternativa ao passo \ref{uc017_p:4} - Alterar marmita \label{uc017_a:3}
	\begin{enumerate}[label*=.\arabic*]
		\item O funcionário aciona a opção \textbf{Alterar}, caso essa funcionalidade esteja habilitada.
		\item O sistema executa o caso de uso \nameref{uc005}.
		\item Esse caso de uso é encerrado.
	\end{enumerate}
	
	\item Alternativa ao passo \ref{uc017_p:4} - Excluir marmita \label{uc017_a:4}
	\begin{enumerate}[label*=.\arabic*]
		\item O funcionário aciona a opção \textbf{Excluir}, caso essa funcionalidade esteja habilitada.
		\item O sistema remove os dados da marmita. \label{uc017_a:4:2}\ref{uc017_e:1}
		\item O sistema volta à tela anterior.
		\item Esse caso de uso é encerrado.
	\end{enumerate}
\end{enumerate}

\subsection{Exceções}

\begin{enumerate}[label=E\arabic*]
	\item Marmita tem pedido vinculado \label{uc017_e:1}
	\begin{enumerate}[label*=.\arabic*]
		\item[] No passo \ref{uc017_a:4:2}, o funcionário tentou remover a marmita, mas isso não foi possível, por que essa tinha pelo menos um pedido vinculado a mesma.
		\item O sistema exibe a mensagem \textbf{A marmita tem pedido vinculado}.
		\item O sistema retorna ao passo \ref{uc017_p:1}.
	\end{enumerate}
\end{enumerate}

\section{Pós-condições}

\begin{enumerate}
	\item O sistema exibirá os dados da marmita.	
\end{enumerate}

\section{Regras de negócios especiais}

\begin{enumerate}[label=ED\arabic*]
	\item Exibe os campos de dados da marmita de acordo com a tabela \ref{uc017_tb_rn1}. \label{uc017_rn:1}
	\begin{table}[htb]
		\ABNTEXfontereduzida
		\caption[Campos de dados da marmita]{Campos de dados da marmita.}
		\label{uc017_tb_rn1}
		\begin{tabular}{|p{4.0cm}|p{3.0cm}|p{7.25cm}|}
			\hline
			\textbf{Campo}  & \textbf{Tipo} & \textbf{Observação}                                                   \\ \hline
			Cód. do produto & Integer       & N.A                                                                   \\ \hline
			Data cadastro   & Date          & O sistema exibirá a data de cadastro no formato \textbf{dia/mês/ano}. \\ \hline
			Nome            & String        & N.A                                                                   \\ \hline
			Ingrediente     & String        & N.A                                                                   \\ \hline
			Custo           & Float         & O sistema exibirá o custo da marmita no formato moeda.                \\ \hline
			Tamanho           & Enum          & O sistema exibirá uma das seguintes opções: 	
			\begin{enumerate}
				\item GRANDE (tamanho grande);
				\item MEDIO (tamanho médio);
				\item PEQUENO (tamanho pequeno).
			\end{enumerate}\\ \hline
		\end{tabular}
	\end{table}
\end{enumerate}