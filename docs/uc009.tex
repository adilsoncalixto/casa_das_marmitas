\chapter{Realizar Entrega - UC009} \label{uc009}

\section{Breve descrição}

Os pedidos serão atendidos com base nas ligações recebidas e o atendimento ao pedido deverá respeitar a ordem das ligações. Os pedidos ficarão nos seguintes status:

\begin{itemize}
	\item Pendente: Quando o atendente da entrada no pedido.
	\item Em trânsito: Quando o motoboy retira o pedido para a entrega.
	\item Cancelado: Quando surge alguma irregularidade e o pedido não pode ser entregue.
	\item Entregue: Quando o motoboy conclui a entrega e apresenta o pagamento.
\end{itemize}

\section{Atores}

\begin{enumerate}
	\item Atendente da empresa, Casa das Marmitas.
\end{enumerate}

\section{Pré-condições}

\begin{enumerate}
	\item O atendente deverá possuir login e senha de acesso autenticados pelo sistema.
	\item O atendente deverá ter permissão para alterar o status do pedido do cliente.
	\item O atendente deverá ter executado, anteriormente, o caso de uso \nameref{uc013}.
	\item O atendente deverá ter selecionado o pedido na lista de pedidos da tela \textbf{Exibir Cliente}.
\end{enumerate}

\section{Fluxo de eventos}

\subsection{Fluxo básico}

\begin{enumerate}[label=P\arabic*]
	\item O sistema apresenta a tela \textbf{Realizar Entrega} com os campos \ref{uc009_rn:1}. \label{uc009_p:1}
	\item No campo \textbf{Status}, o atendente informa a situação do pedido. \label{uc009_p:2}		
	\item O atendente aciona a opção \textbf{Salvar}. \label{uc009_p:3}\ref{uc009_a:1}
	\item O sistema valida os dados dos campos. 
	\item O sistema grava a alteração.
	\item O sistema atualiza a lista de pedidos do cliente na tela \textbf{Exibir Cliente}.
	\item O sistema fecha a tela atual.
	\item O sistema retorna ao caso de uso \nameref{uc013}.
	\item Esse caso de uso é encerrado.	
\end{enumerate}

\subsection{Fluxos alternativos}

\begin{enumerate}[label=A\arabic*]
	\item Alternativa ao passo \ref{uc009_p:3} - Cancelar alteração \label{uc009_a:1}
	\begin{enumerate}[label*=.\arabic*]
		\item O atendente aciona a opção \textbf{Cancelar}.
		\item O sistema fecha a tela atual.
		\item O sistema retorna ao caso de uso \nameref{uc013}.
		\item Esse caso de uso é encerrado.
	\end{enumerate} 
\end{enumerate}

\section{Pós-condições}

\begin{enumerate}
	\item O atendente terá atualizado o status do pedido do cliente.
	\item O sistema terá retornado ao caso de uso \nameref{uc013}.	
\end{enumerate}

\section{Regras de negócios especiais}

\begin{enumerate}[label=RN\arabic*]
	\item Exibe os campos de dados do pedido de acordo com a tabela \ref{uc009_tb_rn1}. \label{uc009_rn:1}
	\begin{table}[htb]
		\ABNTEXfontereduzida
		\caption[Campos de dados do pedido]{Campos de dados do pedido.}
		\label{uc009_tb_rn1}
		\begin{tabular}{|p{3.0cm}|p{2.0cm}|p{1.5cm}|p{2.0cm}|p{5.75cm}|}
			\hline
			\textbf{Campo}   & \textbf{Tipo} & \textbf{Tamanho} & \textbf{Obrigatório} & \textbf{Observação}                                                                                                                                              \\ \hline
			Cód. do pedido   & Long          & N.A & SIM & N.A                                                                                                                                                              \\ \hline
			Data cadastro    & Date          & N.A & SIM & O sistema exibirá a data de cadastro no formato \textbf{dia/mês/ano hora:minuto}.                                                                                \\ \hline
			Quantidade total & Short         & N.A & SIM & N.A                                                                                                                                                              \\ \hline
			Total do pedido  & Float         & N.A & SIM & O sistema exibirá, no formato moeda, o valor total do pedido.                                                                                                    \\ \hline
			Status           & Enum          & N.A & SIM & O atendente terá que informar uma das seguintes opções: 	
			\begin{enumerate}
				\item PENDENTE (pedido pendente);
				\item TRANSITO (pedido em trânsito);
				\item CANCELADO (pedido cancelado);
				\item ENTREGUE (pedido entregue).
			\end{enumerate}\\ \hline
		\end{tabular}
	\end{table}
\end{enumerate}