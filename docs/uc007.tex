\chapter{Registrar Pedido - UC007} \label{uc007}

\section{Breve descrição}

Para gerenciar os pedidos realizados pelo cliente, o atendente deverá informar os dados do pedido.

O sistema deverá calcular automaticamente o valor total do pedido, por exemplo:

\begin{itemize}
	\item Pedido 1
	\begin{itemize}
		\item Marmita3: R\$ 14,00
		\item Marmita4: R\$ 10,00
		\item SubTotal do pedido: R\$ 24,00
		\item Taxa de entrega: R\$ 4,50
		\item Valor Total do pedido: R\$ 28,50
	\end{itemize}
\end{itemize}

\section{Atores}

\begin{enumerate}
	\item Atendente da empresa, Casa das Marmitas.
\end{enumerate}

\section{Pré-condições}

\begin{enumerate}
	\item O atendente deverá possuir login e senha de acesso autenticados pelo sistema.
	\item O atendente deverá ter permissão para cadastrar pedidos para o cliente.
	\item O atendente deverá ter executado, anteriormente, o caso de uso \nameref{uc013}.
	\item O atendente deverá ter acionado a opção \textbf{Registrar Pedido} da tela \textbf{Exibir Cliente}.
\end{enumerate}

\section{Fluxo de eventos}

\subsection{Fluxo básico}

\begin{enumerate}[label=P\arabic*]
	\item O sistema apresenta a tela \textbf{Registrar Pedido} com os campos \ref{uc007_rn:1}. \label{uc007_p:1}
	\item No campo \textbf{Produto}, o atendente informa o código ou nome de uma marmita. \label{uc007_p:2}
	\item O sistema carrega o caso de uso \nameref{uc012} de acordo com o dado informado no passo anterior.
	\item O atendente seleciona a marmita. \label{uc007_p:4}
	\item O atendente preenche os outros campos da tela. \label{uc007_p:5}	
	\item O atendente aciona a opção \textbf{Incluir item}.
	\item O sistema valida os dados dos campos. \ref{uc007_e:1} \ref{uc007_e:2} \ref{uc007_e:3}
	\item O sistema atualiza a lista de itens de pedido.
	\item O sistema atualiza o valor total do pedido.
	\item O sistema limpa o formulário, com exceção do campo \textbf{Total do pedido}, preenchido pelo atendente.
	\item O atendente aciona a opção \textbf{Salvar}. \label{uc007_a:11}\ref{uc007_a:1} \ref{uc007_a:2} \ref{uc007_a:3}
	\item O sistema registra, com status \textbf{PENDENTE}, o pedido.
	\item O sistema atualiza a lista de pedidos do cliente na tela \textbf{Exibir Cliente}.
	\item O sistema fecha a tela atual.
	\item O sistema retorna ao caso de uso \nameref{uc013}.
	\item Esse caso de uso é encerrado.	
\end{enumerate}

\subsection{Fluxos alternativos}

\begin{enumerate}[label=A\arabic*]
	\item Alternativa ao passo \ref{uc007_a:11} - Incluir novo item de pedido \label{uc007_a:1}
	\begin{enumerate}[label*=.\arabic*]
		\item O atendente volta ao passo \ref{uc007_p:2}.
	\end{enumerate}
	
	\item Alternativa ao passo \ref{uc007_a:11} - Calcular Troco \label{uc007_a:2}
	\begin{enumerate}[label*=.\arabic*]
		\item O atendente aciona a opção \textbf{Calcular Troco}.
		\item O sistema executa o caso de uso \nameref{uc008}.
		\item Esse caso de uso é suspenso.
	\end{enumerate} 
		
	\item Alternativa ao passo \ref{uc007_a:11} - Cancelar inclusão \label{uc007_a:3}
	\begin{enumerate}[label*=.\arabic*]
		\item O atendente aciona a opção \textbf{Cancelar}.
		\item O sistema exibe a mensagem \textbf{Operação cancelada}.
		\item O atendente aciona a opção \textbf{Ok}.
		\item O sistema fecha a tela atual.
		\item O sistema retorna ao caso de uso \nameref{uc013}.
		\item Esse caso de uso é encerrado.
	\end{enumerate} 	
\end{enumerate}

\subsection{Exceções}

\begin{enumerate}[label=E\arabic*]
	\item O atendente não selecionou a marmita \label{uc007_e:1}
	\begin{enumerate}[label*=.\arabic*]
		\item[] No passo \ref{uc007_p:4}, o atendente não selecionou a marmita.
		\item O sistema exibe a mensagem \textbf{Informe a marmita}.
		\item O sistema destaca o campo \textbf{Produto}.
		\item O sistema retorna ao passo \ref{uc007_p:2}.
	\end{enumerate}
	
	\item O atendente não informou algum campo obrigatório \label{uc007_e:2}
	\begin{enumerate}[label*=.\arabic*]
		\item[] No passo \ref{uc007_p:5}, o atendente deixou em branco pelo menos um campo obrigatório.
		\item O sistema exibe a mensagem \textbf{Favor preencher o campo obrigatório}.
		\item O sistema destaca os campos não informados pelo gerente.
		\item O sistema retorna ao passo anterior.
	\end{enumerate}
	
	\item O atendente preencheu de forma errada o campo de quantidade \label{uc007_e:3}
	\begin{enumerate}[label*=.\arabic*]		
		\item[] No passo \ref{uc007_p:5}, o atendente não preencheu de forma correta o campo de quantidade.		
		\item O sistema exibe a mensagem \textbf{Quantidade informada não é válida}.
		\item O sistema destaca o campo \textbf{Quantidade}.
		\item O sistema retorna ao passo anterior.
	\end{enumerate}
\end{enumerate}

\section{Pós-condições}

\begin{enumerate}
	\item O atendente terá cadastrado um novo pedido para o cliente.
	\item O sistema terá retornado ao caso de uso \nameref{uc013}.	
\end{enumerate}

\section{Regras de negócios especiais}

\begin{enumerate}[label=RN\arabic*]
	\item Exibe os campos de dados do item de pedido de acordo com a tabela \ref{uc007_tb_rn1}. \label{uc007_rn:1}
	\begin{table}[htb]
		\ABNTEXfontereduzida
		\caption[Campos de dados do item de pedido]{Campos de dados do item de pedido.}
		\label{uc007_tb_rn1}
		\begin{tabular}{|p{3.0cm}|p{2.0cm}|p{1.5cm}|p{2.0cm}|p{5.75cm}|}
			\hline
			\textbf{Campo}  & \textbf{Tipo} & \textbf{Tamanho} & \textbf{Obrigatório} & \textbf{Observação}                                           \\ \hline
			Produto         & String        & 60               & SIM                  & N.A                                                           \\ \hline
			Tamanho         & Enum          & N.A              & SIM                  & N.A                                                           \\ \hline
			Quantidade      & Short         & N.A              & SIM                  & N.A                                                           \\ \hline
			Total do pedido & Float         & N.A              & SIM                  & O sistema exibirá, no formato moeda, o valor total do pedido. \\ \hline
		\end{tabular}
	\end{table}
\end{enumerate}