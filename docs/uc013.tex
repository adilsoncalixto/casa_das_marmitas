\chapter{Exibir Cliente - UC013} \label{uc013}

\section{Breve descrição}

Após a pesquisa ou o cadastro de um cliente, o sistema carregará um formulário com os dados do mesmo.

\section{Atores}

\begin{enumerate}
	\item Funcionário da empresa, Casa das Marmitas.
\end{enumerate}

\section{Pré-condições}

\begin{enumerate}
	\item O funcionário deverá possuir login e senha de acesso autenticados pelo sistema.
	\item O funcionário deverá ter executado, anteriormente, o caso de uso \nameref{uc001} ou \nameref{uc002}.
\end{enumerate}

\section{Fluxo de eventos}

\subsection{Fluxo básico}

\begin{enumerate}[label=P\arabic*]
	\item O sistema apresenta a tela \textbf{Exibir Cliente} com os campos \ref{uc013_rn:1}. \label{uc013_p:1}
	\item O sistema exibe a lista de pedidos com os campos \ref{uc013_rn:2}. \label{uc013_p:2}\ref{uc013_e:1}
	\item O sistema habilita a opção \textbf{Alterar}. \label{uc013_p:3}\ref{uc013_a:1} 
	\item O sistema habilita a opção \textbf{Excluir}. \label{uc013_p:4}\ref{uc013_a:2}
	\item O sistema habilita opção \textbf{Registrar Pedido}. \label{uc013_p:5}\ref{uc013_a:3}
	\item O sistema habilita a seleção da lista de pedidos. \label{uc013_p:6}\ref{uc013_a:4}
	\item Esse caso de uso é encerrado. \label{uc013_p:7}\ref{uc013_a:5} \ref{uc013_a:6} \ref{uc013_a:7}  \ref{uc013_a:8}
\end{enumerate}

\subsection{Fluxos alternativos}

\begin{enumerate}[label=A\arabic*]
	\item Alternativa ao passo \ref{uc013_p:3} - O funcionário não tem permissão de alterar os dados do cliente \label{uc013_a:1}
	\begin{enumerate}[label*=.\arabic*]
		\item O sistema desabilita a opção \textbf{Alterar}.
		\item O sistema vai para o próximo passo.
	\end{enumerate}

	\item Alternativa ao passo \ref{uc013_p:4} - O funcionário não tem permissão de excluir o cliente \label{uc013_a:2}
	\begin{enumerate}[label*=.\arabic*]
		\item O sistema desabilita a opção \textbf{Excluir}.
		\item O sistema vai para o próximo passo.
	\end{enumerate}

	\item Alternativa ao passo \ref{uc013_p:5} - O funcionário não tem permissão de registrar pedidos para o cliente \label{uc013_a:3}
	\begin{enumerate}[label*=.\arabic*]
		\item O sistema desabilita a opção \textbf{Registrar Pedido}.
		\item O sistema vai para o próximo passo.
	\end{enumerate}
	
	\item Alternativa ao passo \ref{uc013_p:6} - O funcionário não tem permissão de alterar o status dos pedidos do cliente \label{uc013_a:4}
	\begin{enumerate}[label*=.\arabic*]
		\item O sistema desabilita a seleção da lista de pedidos.
		\item O sistema vai para o próximo passo.
	\end{enumerate}
	
	\item Alternativa ao passo \ref{uc013_p:7} - Alterar cliente \label{uc013_a:5}
	\begin{enumerate}[label*=.\arabic*]
		\item O funcionário aciona a opção \textbf{Alterar}, caso essa funcionalidade esteja habilitada.
		\item O sistema executa o caso de uso \nameref{uc001}.
		\item Esse caso de uso é encerrado.
	\end{enumerate}

	\item Alternativa ao passo \ref{uc013_p:7} - Excluir cliente \label{uc013_a:6}
	\begin{enumerate}[label*=.\arabic*]
		\item O funcionário aciona a opção \textbf{Excluir}, caso essa funcionalidade esteja habilitada.
		\item O sistema remove os dados do cliente. \label{uc013_a:6:2}\ref{uc013_e:2}
		\item O sistema volta à tela anterior.
		\item Esse caso de uso é encerrado.
	\end{enumerate}

	\item Alternativa ao passo \ref{uc013_p:7} - Registrar pedido \label{uc013_a:7}
	\begin{enumerate}[label*=.\arabic*]
		\item O funcionário aciona a opção \textbf{Registrar Pedido}, caso essa funcionalidade esteja habilitada.
		\item O sistema executa o caso de uso \nameref{uc007}.
		\item Esse caso de uso é suspenso.
	\end{enumerate}
	
	\item Alternativa ao passo \ref{uc013_p:7} - Alterar status do pedido \label{uc013_a:8}
	\begin{enumerate}[label*=.\arabic*]
		\item O funcionário seleciona o pedido, caso essa funcionalidade esteja habilitada.
		\item O sistema executa o caso de uso \nameref{uc009}.
		\item Esse caso de uso é suspenso.
	\end{enumerate}
\end{enumerate}

\subsection{Exceções}

\begin{enumerate}[label=E\arabic*]
	\item Cliente não tem pedido cadastrado \label{uc013_e:1}
	\begin{enumerate}[label*=.\arabic*]
		\item[] No passo \ref{uc013_p:2}, não foi encontrado nenhum pedido vinculado à conta do cliente.
		\item O sistema exibe a mensagem \textbf{O cliente não tem pedidos cadastrados}.
		\item O sistema vai para o próximo passo.
	\end{enumerate}

	\item Cliente tem pedido registrado \label{uc013_e:2}
	\begin{enumerate}[label*=.\arabic*]
		\item[] No passo \ref{uc013_a:6:2}, o funcionário tentou remover o cliente, mas isso não foi possível, por que esse tinha pelo menos um pedido cadastrado.
		\item O sistema exibe a mensagem \textbf{O cliente tem pedido registrado}.
		\item O sistema retorna ao passo \ref{uc013_p:1}.
	\end{enumerate}
\end{enumerate}

\section{Pós-condições}

\begin{enumerate}
	\item O sistema exibirá os dados do cliente.	
\end{enumerate}

\section{Regras de negócios especiais}

\begin{enumerate}[label=RN\arabic*]
	\item Exibe os campos de dados do cliente de acordo com a tabela \ref{uc013_tb_rn1}. \label{uc013_rn:1}
	\begin{table}[htb]
		\ABNTEXfontereduzida
		\caption[Campos de dados do cliente]{Campos de dados do cliente.}
		\label{uc013_tb_rn1}
		\begin{tabular}{|p{4.0cm}|p{3.0cm}|p{7.25cm}|}
			\hline
			\textbf{Campo}      & \textbf{Tipo} & \textbf{Observação}                                                        \\ \hline
			Cód. do cliente     & Integer       & N.A                                                                        \\ \hline
			Data cadastro       & Date          & O sistema exibirá a data de cadastro no formato \textbf{dia/mês/ano}.      \\ \hline
			Nome                & String        & N.A                                                                        \\ \hline
			Data de nascimento  & Date          & O sistema exibirá a data de nascimento no formato \textbf{dia/mês/ano}.    \\ \hline
			Telefone            & String        & O sistema exibirá o número de telefone no formato \textbf{(99) 9999-9999}. \\ \hline
			Logradouro          & String        & N.A                                                                        \\ \hline
			CEP                 & String        & O sistema exibirá o CEP no formato \textbf{99.999-999}.                    \\ \hline
			Bairro              & String        & N.A                                                                        \\ \hline
			Cidade              & String        & N.A                                                                        \\ \hline
			Número              & String        & N.A                                                                        \\ \hline
			Complemento         & String        & N.A                                                                        \\ \hline
			Ponto de referência & String        & N.A                                                                        \\ \hline
		\end{tabular}
	\end{table}
	
	\item Exibe os campos de dados do pedido de acordo com a tabela \ref{uc013_tb_rn2}. \label{uc013_rn:2}
	\begin{table}[htb]
		\ABNTEXfontereduzida
		\caption[Campos de dados do pedido]{Campos de dados do pedido.}
		\label{uc013_tb_rn2}
		\begin{tabular}{|p{4.0cm}|p{3.0cm}|p{7.25cm}|}
			\hline
			\textbf{Campo}   & \textbf{Tipo} & \textbf{Observação}                                                                                                                                              \\ \hline
			Cód. do pedido   & Long          & N.A                                                                                                                                                              \\ \hline
			Data cadastro    & Date          & O sistema exibirá a data de cadastro no formato \textbf{dia/mês/ano hora:minuto}.                                                                                \\ \hline
			Quantidade total & Short         & N.A                                                                                                                                                              \\ \hline
			Total do pedido  & Float         & O sistema exibirá, no formato moeda, o valor total do pedido.                                                                                                    \\ \hline
			Status           & Enum          & O sistema exibirá uma das seguintes opções: 
			\begin{enumerate}
				\item PENDENTE (pedido pendente);
				\item TRANSITO (pedido em trânsito);
				\item CANCELADO (pedido cancelado);
				\item ENTREGUE (pedido entregue).
			\end{enumerate}\\ \hline
		\end{tabular}
	\end{table}
\end{enumerate}