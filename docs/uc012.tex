\chapter{Pesquisa Produto - UC012} \label{uc012}

\section{Breve descrição}

O sistema exibirá as marmitas que foram pesquisadas pelo funcionário.

\section{Atores}

\begin{enumerate}
	\item Funcionário da empresa, Casa das Marmitas.
\end{enumerate}

\section{Pré-condições}

\begin{enumerate}
	\item O funcionário deverá possuir login e senha de acesso autenticados pelo sistema.
	\item O funcionário deverá ter permissão para realizar a busca de marmitas.
\end{enumerate}

\section{Fluxo de eventos}

\subsection{Fluxo básico}

\begin{enumerate}[label=P\arabic*]
	\item O funcionário aciona a opção \textbf{Marmita >> Listar} no menu do sistema. \label{uc012_p:1}\ref{uc012_a:1}
	\item O sistema apresenta a tela \textbf{Lista de Marmitas}.	
	\item O funcionário informa, no campo de pequisa, o código ou nome da marmita. \label{uc012_p:3}
	\item O funcionário seleciona a opção \textbf{Buscar}.
	\item O sistema exibe a lista de marmitas com os campos \ref{uc012_rn:1}. \ref{uc012_e:1}
	\item O funcionário seleciona a marmita.
	\item O sistema executa o caso de uso \nameref{uc017}, de acordo com a marmita selecionada no passo anterior.
	\item Esse caso de uso é encerrado.
\end{enumerate}

\subsection{Fluxos alternativos}

\begin{enumerate}[label=A\arabic*]
	\item Alternativa ao passo \ref{uc012_p:1} - Relacionar marmita ao pedido \label{uc012_a:1}
	\begin{enumerate}[label*=.\arabic*]
		\item Na tela fornecida pelo sistema através do caso de uso \nameref{uc007}, o funcionário preenche o campo \textbf{Produto}. \label{uc012_a:1:1}
		\item O sistema exibe a lista de marmitas, em uma tela, com os campos \ref{uc012_rn:1}. \ref{uc012_e:1}
		\item O funcionário seleciona a marmita. \label{uc012_a:1:3}
		\item O sistema atualiza o campo \textbf{Produto} na tela \textbf{Registrar Pedido}.
		\item O sistema fecha a tela atual.
		\item O sistema retorna ao caso de uso \nameref{uc007}.
		\item Esse caso de uso é encerrado.
	\end{enumerate}
	
	\item Alternativa ao passo \ref{uc012_a:1:3} - Cancelar seleção
	\begin{enumerate}[label*=.\arabic*]
		\item O funcionário aciona a opção \textbf{Cancelar}.
		\item O sistema fecha a tela atual.
		\item O sistema retorna ao caso de uso \nameref{uc007}.
		\item Esse caso de uso é encerrado.
	\end{enumerate}
\end{enumerate}

\subsection{Exceções}

\begin{enumerate}[label=E\arabic*]
	\item Marmita não cadastrada \label{uc012_e:1}
	\begin{enumerate}[label*=.\arabic*]
		\item[] No passo \ref{uc012_p:3} ou \ref{uc012_a:1:1}, o funcionário informou o código ou nome de uma marmita não cadastrada no sistema.
		\item O sistema exibe a mensagem \textbf{Marmita não cadastrada}.
		\item O sistema retorna ao passo anterior.
	\end{enumerate}
\end{enumerate}

\section{Pós-condições}

\begin{itemize}
	\item Marmita selecionada na tela \textbf{Lista de Marmitas}
	\begin{enumerate}
		\item O sistema executará o caso de uso \nameref{uc017}.	
	\end{enumerate}

	\item Marmita selecionada através da tela \textbf{Registrar Pedido}
	\begin{enumerate}
		\item O sistema carregará, no campo \textbf{Produto}, o nome da marmita. 
	\end{enumerate}
\end{itemize}

\section{Regras de negócios especiais}

\begin{enumerate}[label=ED\arabic*]
	\item Exibe os campos de dados da marmita de acordo com a tabela \ref{uc012_tb_rn1}. \label{uc012_rn:1}
	\begin{table}[htb]
		\ABNTEXfontereduzida
		\caption[Campos de dados da marmita]{Campos de dados da marmita.}
		\label{uc012_tb_rn1}
		\begin{tabular}{|p{4.0cm}|p{3.0cm}|p{7.25cm}|}
			\hline
			\textbf{Campo}  & \textbf{Tipo} & \textbf{Observação}                                                   \\ \hline
			Cód. do produto & Integer       & N.A                                                                   \\ \hline
			Data cadastro   & Date          & O sistema exibirá a data de cadastro no formato \textbf{dia/mês/ano}. \\ \hline
			Nome            & String        & N.A                                                                   \\ \hline			
			Custo           & Float         & O sistema exibirá o custo da marmita no formato moeda.                \\ \hline
			Tamanho           & Enum          & O sistema exibirá uma das seguintes opções: 	
			\begin{enumerate}
				\item GRANDE (tamanho grande);
				\item MEDIO (tamanho médio);
				\item PEQUENO (tamanho pequeno).
			\end{enumerate}\\ \hline
		\end{tabular}
	\end{table}
\end{enumerate}