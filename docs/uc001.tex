\chapter{Manter Cliente - UC001} \label{uc001}
 
\section{Breve descrição}
 
O sistema permitirá o cadastro e a alteração dos dados do cliente.

\section{Atores}

\begin{enumerate}
	\item Funcionário da empresa, Casa das Marmitas.
\end{enumerate}

\section{Pré-condições}

\begin{enumerate}
	\item O funcionário deverá possuir login e senha de acesso autenticados pelo sistema.
	\item O funcionário deverá ter permissão para realizar o cadastro e a alteração dos dados do cliente.
\end{enumerate}

\section{Fluxo de eventos}

\subsection{Fluxo básico}

\begin{enumerate}[label=P\arabic*]
	\item O funcionário aciona a opção \textbf{Cliente >> Incluir} no menu do sistema. \label{uc001_p:1}\ref{uc001_a:1}
	\item O sistema apresenta a tela \textbf{Incluir Cliente} com os campos \ref{uc001_rn:1}. \label{uc001_p:2}
	\item O funcionário preenche os campos da tela. \label{uc001_p:3}
	\item O funcionário aciona a opção \textbf{Salvar}. \label{uc001_p:4}\ref{uc001_a:2}
	\item O sistema valida os dados dos campos. \ref{uc001_e:1} \ref{uc001_e:2} \ref{uc001_e:3} \ref{uc001_e:4} \ref{uc001_e:5} \ref{uc001_e:6} \ref{uc001_e:7}
	\item O sistema realiza a inclusão com sucesso.
	\item O sistema executa o caso de uso \nameref{uc013}.
	\item Esse caso de uso é encerrado.	
\end{enumerate}

\subsection{Fluxos alternativos}

\begin{enumerate}[label=A\arabic*]
	\item Alternativa ao passo \ref{uc001_p:1} - Alterar cliente \label{uc001_a:1}
	\begin{enumerate}[label*=.\arabic*]
		\item Na tela fornecida pelo sistema através do caso de uso \nameref{uc013}, o funcionário aciona a opção \textbf{Alterar}. 
		\item O sistema apresenta a tela \textbf{Alterar Cliente} com os campos \ref{uc001_rn:1}. \label{uc001_a:1:2}
		\item O funcionário preenche os campos da tela. \label{uc001_a:1:3}
		\item O funcionário aciona a opção \textbf{Salvar}. \label{uc001_a:1:4}\ref{uc001_a:2}
		\item O sistema valida os dados dos campos. \ref{uc001_e:1} \ref{uc001_e:2} \ref{uc001_e:3} \ref{uc001_e:4} \ref{uc001_e:5} \ref{uc001_e:6}
		\item O sistema altera os dados com sucesso.
		\item O sistema executa o caso de uso \nameref{uc013}.
		\item Esse caso de uso é encerrado.
	\end{enumerate}

	\item Alternativa ao passo \ref{uc001_p:4} ou \ref{uc001_a:1:4} - Cancelar inclusão ou alteração \label{uc001_a:2}
	\begin{enumerate}[label*=.\arabic*]
		\item O funcionário aciona a opção \textbf{Cancelar}.
		\item O sistema exibe a mensagem \textbf{Operação cancelada}.
		\item O funcionário aciona a opção \textbf{Ok}.
		\item O sistema retorna à tela principal.
		\item Esse caso de uso é encerrado.
	\end{enumerate}			 	
\end{enumerate}

\subsection{Exceções}

\begin{enumerate}[label=E\arabic*]
	\item O funcionário não informou algum campo obrigatório \label{uc001_e:1}
	\begin{enumerate}[label*=.\arabic*]
		\item[] No passo \ref{uc001_p:3} ou \ref{uc001_a:1:3}, o funcionário deixou em branco pelo menos um campo obrigatório.
		\item O sistema exibe a mensagem \textbf{Favor preencher o campo obrigatório}.
		\item O sistema destaca os campos não informados pelo funcionário.
		\item O sistema retorna ao passo anterior.
	\end{enumerate}

	\item O funcionário preencheu de forma errada o campo da data de nascimento \label{uc001_e:2}
	\begin{enumerate}[label*=.\arabic*]
			\item[] No passo \ref{uc001_p:3} ou \ref{uc001_a:1:3}, o funcionário não preencheu de forma correta o campo da data de nascimento, por exemplo, usou somente letras.				
		\item O sistema exibe a mensagem \textbf{A data de nascimento informada não é válida}.
		\item O sistema destaca o campo \textbf{Data de nascimento}.
		\item O sistema retorna ao passo anterior.
	\end{enumerate}

	\item O funcionário preencheu de forma errada o campo de telefone \label{uc001_e:3}
	\begin{enumerate}[label*=.\arabic*]		
		\item[] No passo \ref{uc001_p:3} ou \ref{uc001_a:1:3}, o funcionário não preencheu de forma correta o campo de telefone, usando letras ou qualquer outro carácter diferente de número.		
		\item O sistema exibe a mensagem \textbf{O número de telefone informado não é válido}.
		\item O sistema destaca o campo \textbf{Telefone}.
		\item O sistema retorna ao passo anterior.
	\end{enumerate}

	\item O funcionário preencheu de forma errada o campo de logradouro \label{uc001_e:4}
	\begin{enumerate}[label*=.\arabic*]		
		\item[] No passo \ref{uc001_p:3} ou \ref{uc001_a:1:3}, o funcionário não preencheu de forma correta o campo de logradouro (rua, avenida, beco, ...), por exemplo, usou somente números.		
		\item O sistema exibe a mensagem \textbf{O nome do logradouro informado não é válido}.
		\item O sistema destaca o campo \textbf{Logradouro}.
		\item O sistema retorna ao passo anterior.
	\end{enumerate}

	\item O funcionário preencheu de forma errada o campo de CEP \label{uc001_e:5}
	\begin{enumerate}[label*=.\arabic*]		
		\item[] No passo \ref{uc001_p:3} ou \ref{uc001_a:1:3}, o funcionário não preencheu de forma correta o campo de CEP, usando letras ou menos de 8 dígitos.		
		\item O sistema exibe a mensagem \textbf{O CEP informado não é válido}.
		\item O sistema destaca o campo \textbf{CEP}.
		\item O sistema retorna ao passo anterior.
	\end{enumerate}

	\item O funcionário preencheu de forma errada o campo de cidade \label{uc001_e:6}
	\begin{enumerate}[label*=.\arabic*]		
		\item[] No passo \ref{uc001_p:3} ou \ref{uc001_a:1:3}, o funcionário não preencheu de forma correta o campo de cidade, por exemplo, usou somente números.		
		\item O sistema exibe a mensagem \textbf{O nome da cidade informado não é válido}.
		\item O sistema destaca o campo \textbf{Cidade}.
		\item O sistema retorna ao passo anterior.
	\end{enumerate}

	\item Cliente cadastrado anteriormente \label{uc001_e:7}
	\begin{enumerate}[label*=.\arabic*]
		\item[] No passo \ref{uc001_p:3} do fluxo básico, o funcionário preencheu os dados de um cliente já cadastrado no sistema.
		\item O sistema exibe a mensagem \textbf{O cliente informado já foi cadastrado anteriormente no sistema}.
		\item O sistema retorna ao passo anterior.
	\end{enumerate}
\end{enumerate}

\section{Pós-condições}

\begin{enumerate}
	\item O funcionário terá cadastrado ou alterado os dados do cliente.
	\item O sistema executará o caso de uso \nameref{uc013}.	
\end{enumerate}

\section{Regras de negócios especiais}

\begin{enumerate}[label=RN\arabic*]
	\item Exibe os campos de dados do cliente de acordo com a tabela \ref{uc001_tb_rn1}. \label{uc001_rn:1}
	\begin{table}[htb]
		\ABNTEXfontereduzida
		\caption[Campos de dados do cliente]{Campos de dados do cliente.}
		\label{uc001_tb_rn1}
		\begin{tabular}{|p{3.0cm}|p{2.0cm}|p{1.5cm}|p{2.0cm}|p{5.75cm}|}
			\hline
			\textbf{Campo}      & \textbf{Tipo} & \textbf{Tamanho} & \textbf{Obrigatório} & \textbf{Observação}                                                                      \\ \hline
			Nome                & String        & 60               & SIM                  & N.A                                                                                      \\ \hline
			Data de nascimento  & Date          & N.A              & NÃO                  & O funcionário terá que informar a data de nascimento no formato \textbf{dia/mês/ano}.    \\ \hline
			Telefone            & String        & 10               & SIM                  & O funcionário terá que informar o número de telefone no formato \textbf{(99) 9999-9999}. \\ \hline
			Logradouro          & String        & 100              & SIM                  & N.A                                                                                      \\ \hline
			CEP                 & String        & 8                & SIM                  & O funcionário terá que informar o CEP no formato \textbf{99.999-999}.                    \\ \hline
			Bairro              & String        & 60               & SIM                  & N.A                                                                                      \\ \hline
			Cidade              & String        & 60               & SIM                  & N.A                                                                                      \\ \hline
			Número              & String        & 30               & SIM                  & N.A                                                                                      \\ \hline
			Complemento         & String        & 30               & NÃO                  & N.A                                                                                      \\ \hline
			Ponto de referência & String        & 30               & NÃO                  & N.A                                                                                      \\ \hline
		\end{tabular}
	\end{table}
\end{enumerate}